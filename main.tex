%%
%% This is file `sample-sigconf.tex',
%% generated with the docstrip utility.
%%
%% The original source files were:
%%
%% samples.dtx  (with options: `sigconf')
%% 
%% IMPORTANT NOTICE:
%% 
%% For the copyright see the source file.
%% 
%% Any modified versions of this file must be renamed
%% with new filenames distinct from sample-sigconf.tex.
%% 
%% For distribution of the original source see the terms
%% for copying and modification in the file samples.dtx.
%% 
%% This generated file may be distributed as long as the
%% original source files, as listed above, are part of the
%% same distribution. (The sources need not necessarily be
%% in the same archive or directory.)
%%
%% The first command in your LaTeX source must be the \documentclass command.
\documentclass[sigconf]{acmart}

\usepackage{code}
\hypersetup{draft}
%%
%% \BibTeX command to typeset BibTeX logo in the docs
\AtBeginDocument{%
  \providecommand\BibTeX{{%
    \normalfont B\kern-0.5em{\scshape i\kern-0.25em b}\kern-0.8em\TeX}}}

%% Rights management information.  This information is sent to you
%% when you complete the rights form.  These commands have SAMPLE
%% values in them; it is your responsibility as an author to replace
%% the commands and values with those provided to you when you
%% complete the rights form.
\setcopyright{acmcopyright}
\copyrightyear{2021}
\acmYear{2021}
\acmDOI{10.1145/1122445.1122456}

%% These commands are for a PROCEEDINGS abstract or paper.
\acmConference[Woodstock '18]{Woodstock '18: ACM Symposium on Neural
  Gaze Detection}{June 03--05, 2018}{Woodstock, NY}
\acmBooktitle{Woodstock '18: ACM Symposium on Neural Gaze Detection,
  June 03--05, 2018, Woodstock, NY}
\acmPrice{15.00}
\acmISBN{978-1-4503-XXXX-X/18/06}


%%
%% Submission ID.
%% Use this when submitting an article to a sponsored event. You'll
%% receive a unique submission ID from the organizers
%% of the event, and this ID should be used as the parameter to this command.
%%\acmSubmissionID{123-A56-BU3}

%%
%% The majority of ACM publications use numbered citations and
%% references.  The command \citestyle{authoryear} switches to the
%% "author year" style.
%%
%% If you are preparing content for an event
%% sponsored by ACM SIGGRAPH, you must use the "author year" style of
%% citations and references.
%% Uncommenting
%% the next command will enable that style.
%%\citestyle{acmauthoryear}

%%
%% end of the preamble, start of the body of the document source.
\begin{document}

%%
%% The "title" command has an optional parameter,
%% allowing the author to define a "short title" to be used in page headers.
\title{DeepState: Unit Testing Unbound}

%%
%% The "author" command and its associated commands are used to define
%% the authors and their affiliations.
%% Of note is the shared affiliation of the first two authors, and the
%% "authornote" and "authornotemark" commands
%% used to denote shared contribution to the research.
\author{Alex Groce}
\affiliation{\institution{Northern Arizona University}\country{United States}}
\author{Peter Goodman}
\author{Gustavo Grieco}
\affiliation{\institution{Trail of Bits}\country{United States}}
\author{Alan Cao}
\affiliation{\institution{New York University}\country{United States}}



%%
%% By default, the full list of authors will be used in the page
%% headers. Often, this list is too long, and will overlap
%% other information printed in the page headers. This command allows
%% the author to define a more concise list
%% of authors' names for this purpose.
\renewcommand{\shortauthors}{Groce, Goodman, Grieco, and Cao}

%%
%% The abstract is a short summary of the work to be presented in the
%% article.
\begin{abstract}
Almost every software developer knows how to write a unit test; very few software developers know how to use fuzzing or symboilic execution tools.  DeepState provides a way to write parameterized/generalized unit tests by adding data generation and nondeterministic choice constructs to a Google Test-like API.  DeepState can then use modern fuzzers such as afl or libFuzzer, or the Manticore symbolic execution tool, to construct concrete tests.  DeepState makes it easy to apply mulitple fuzzers, including in a cooperative ensemble, to a testing problem, and to use fuzzers for full-fledged property-based testing with complex input validity and correctness constraints.  DeepState also adds swarm testing and smart test reduction and normalization capabilities to fuzzers without the need to modify the back-end tool.
\end{abstract}

\begin{CCSXML}
<ccs2012>
<concept>
<concept_id>10011007.10010940.10010992.10010998.10011001</concept_id>
<concept_desc>Software and its engineering~Dynamic analysis</concept_desc>
<concept_significance>500</concept_significance>
</concept>
<concept>
<concept_id>10011007.10011074.10011099.10011102.10011103</concept_id>
<concept_desc>Software and its engineering~Software testing and debugging</concept_desc>
<concept_significance>500</concept_significance>
</concept>
</ccs2012>
\end{CCSXML}

\ccsdesc[500]{Software and its engineering~Dynamic analysis}
\ccsdesc[500]{Software and its engineering~Software testing and debugging}

\keywords{parameterized unit tests, fuzzing, symbolic execution, test reduction}


\maketitle

\section{Introduction}

Automated test generation using sophisticated methods, including fuzzers and symbolic execution, has the potential to greatly improve the reliability, safety, and security of software systems.  Unfortunately, few real-world developers have any experience using such tools, and the widely differing ways test harnesses must be written \cite{WODACommon} and tools must be run limit knowledge transfer from one tool to another.  However, a great many developers \emph{do} know how to write \emph{unit tests} and use a unit testing framework.  For example, Google's GoogleTest is used in Chromium, LLVM, OpenCV, and many other major open source projects, and GoogleTest is referenced by more than 1 million GitHub repositories!

DeepState \cite{goodman2018deepstate} provides a single front-end that allows a developer to \emph{write a single parameterized unit test} \cite{ParamUnit}  and then use a variety of fuzzers and/or symbolic execution tools to generate test cases in a single, shared, binary format.  A parameterized (or generalized) unit test is a unit test that has ``holes'' -- unknown values to be filled in by a tool.  Accordingly, of course, properties must be expressed as a function of the values, rather than fixed, as in \emph{property-based testing} \cite{ClaessenH00}.  The goal of DeepState is to enable developers to make use of modern test generation tools, especially including coverage-driven fuzzers, without having to learn a new paradigm, or restrict their testing to finding crashes.  With DeepState, unit tests are truly ``unbound'' from fixed concrete values, and can be used to find problems humans would almost certainly never discover using manually written unit tests.

DeepState has been used in internal security audits at Trail of Bits, and to detect bugs in real-world code (\url{https://github.com/Blosc/c-blosc2/issues/95}, \url{https://github.com/laurynas-biveinis/unodb}, \url{https://github.com/RoaringBitmap/CRoaring}), including with a tool for fuzzing R libraries written using the popular Rcpp framework (\url{https://github.com/akhikolla/RcppDeepState}); DeepState is being used in various non-public testing efforts, as well.

\section{Why Unit Tests Need DeepState}

The effort to use DeepState to test the popular C-Blosc (\url{https://github.com/Blosc/c-blosc}, \url{https://github.com/Blosc/c-blosc2}) compression library shows the value of using ``unbound'' parameterized unit tests.  C-Blosc includes extensive tests, including tests checking random data for various round-trip properties.  DeepState makes it possible to more succinctly express many of these tests: https://github.com/agroce/deepstate-c-blosc2.

Moreover, DeepState's use of modern fuzzers made it possible to discover a subtle bug (since fixed, see \url{https://github.com/Blosc/c-blosc2/commit/24368ccbcd32c9d3ee6bcafa04c873b7e6751f78}) in C-Blosc2, the successor to the original C-Blosc, \url{https://github.com/Blosc/c-blosc2/issues/95}.  The fully minimized and normalized test (see below) shows that if C-Blosc2 is given a buffer of size 131, consisting of all 0 bytes except for byte 128 (which is 1), then when C-Blosc2 is requested to return the 129th byte without fully decompressing the buffer, it will return a value of 1 instead of 0.  No smaller or simpler input will expose the bug.

\section{A Simple Example}

\begin{figure}
{\scriptsize
\begin{code}
\#include <deepstate/DeepState.hpp>
using namespace deepstate;

\#include  "runlen.h"

// This can be much higher for fuzzing, but should be small for
// symbolic execution to scale
\#define MAX\_STR\_LEN 6

TEST(Runlength, EncodeDecode) \{
  char* original = DeepState\_CStrUpToLen(MAX\_STR\_LEN,
                                         "abcdef0123456789");
  char* encoded = encode(original);
  ASSERT\_LE(strlen(encoded), strlen(original)*2) <<
   "Encoding is > length*2!";
  char* roundtrip = decode(encoded);
  ASSERT\_EQ(strncmp(roundtrip, original, MAX\_STR\_LEN), 0) <<
    "ORIGINAL: '" << original << "', ENCODED: '" << encoded <<
    "', ROUNDTRIP: '" << roundtrip << "'";
\}
\end{code}
}
  \caption{A Simple DeepState Test Harness}
  \label{fig:example}
  \end{figure}

As Figure~\ref{fig:example}, taken from an example in the DeepState repository, but omitting the actual (buggy) code for a runlength encoding shows, DeepState test harnesses look very much like unit tests using the very widely used GoogleTest (\url{https://github.com/google/googletest}) framework.  The critical difference is the use of DeepState API calls to replace fixed, concrete values with tool-generated values, and the use of more general, property-based-testing style, specifications to account for the variability of test values.  Here, the ``work'' is done by {\tt DeepState\_CStrUpToLen}, which takes a maximum string length and (optionally) an alphabet to use, and generates (and manages memory for) string inputs.  DeepState similarly provides APIs for all common C/C++ base types, including generation of values in ranges (e.g., {\tt DeepState\_IntInRange}, {\tt DeepState\_DoubleInRange}, etc.).
Once such a harness is written, test generation is easy.  Simply compiling the test and linking with the DeepState binary, via a command such as {\tt clang++ -o runlen Runlen.cpp -ldeepstate} allows a user to use symbolic execution via Manticore \cite{mossberg2019manticore}, the Eclipser fuzzer \cite{choi:icse:2019}, afl in QEMU mode, or DeepState's built-in dumb (but very fast) fuzzer (Figure~\ref{fig:rundeep}).

\begin{figure}
  {\scriptsize
\begin{code}
> clang++ -o runlen Runlen.cpp -ldeepstate
> deepstate-manticore ./runlen --output\_test\_dir manticore\_symex
INFO:deepstate:Setting log level from DEEPSTATE\_LOG: 2
INFO:deepstate.core.base:Setting log level from --min\_log\_level: 2
2021-04-27 19:16:51,953: [24] m.n.manticore:INFO: Loading program ./runlen
2021-04-27 19:21:37,415: [24] m.n.manticore:INFO: Loading program ./runlen
...
INFO:deepstate.core.symex:Saved test case in file
manticore\_symex/Runlen.cpp/Runlength\_EncodeDecode/
a2604eeb87f7be3ff2990e63e02bdeed.crash
...
INFO:deepstate.core.symex:Saved test case in file
manticore\_symex/Runlen.cpp/Runlength\_EncodeDecode/
7d737b2217daf7671e3346b2cf533270.crash
...

...
> deepstate-eclipser ./runlen --output\_test\_dir eclipser\_fuzzing --fuzzer\_out
INFO:deepstate:Setting log level from DEEPSTATE\_LOG: 2
INFO:deepstate.core.base:Setting log level from --min\_log\_level: 2
INFO:deepstate.core.fuzz:Calling pre\_exec before fuzzing
WARNING:deepstate.executors.fuzz.eclipser:Eclipser doesn't limit child
processes memory.
INFO:deepstate.core.fuzz:Executing command
`['dotnet', '/home/user/deps/eclipser/Eclipser.dll', 'fuzz', '--program',
'/home/user/deepstate/examples/runlen', '--src', 'file', '--fixfilepath',
  'eclipser.input', '--initarg', '--input\_test\_file eclipser.input
  --abort\_on\_fail --no\_fork --min\_log\_level 2', '--outputdir',
  'eclipser\_fuzzing/the\_fuzzer', '--maxfilelen', '8192', '--timelimit',
  '99999']`
INFO:deepstate.core.fuzz:Using fuzzer output.
[00:00:00:00] [*] Fuzz target : /home/user/deepstate/examples/runlen
[00:00:00:00] [*] Time limit : 99999 sec
[00:00:00:00] [*] Total 1 initial seeds
[00:00:00:00] [*] 1 initial items with high priority
[00:00:00:00] [*] 0 initial items with low priority
[00:00:00:00] [*] Fuzzing starts
[00:00:00:00] [*] Found by grey-box concolic (9 new edges) 
[00:00:00:00] Save crash seed : ("--input\_test\_file" "eclipser.input"
  "--abort\_on\_fail" "--no\_fork" "--min\_log\_level" "2")
  File[0]=( 96* ...15bytes... (0) (Right))
[00:00:00:00] [*] Found by grey-box concolic (139 new edges) 
[00:00:00:00] Save crash seed : ("--input\_test\_file" "eclipser.input"
  "--abort\_on\_fail" "--no\_fork" "--min\_log\_level" "2")
  File[0]=( 7d* ...15bytes... (0) (Right))
[00:00:00:00] [*] Found by grey-box concolic (13 new edges) 
[00:00:00:00] Save crash seed : ("--input\_test\_file" "eclipser.input"
  "--abort\_on\_fail" "--no\_fork" "--min\_log\_level" "2")
  File[0]=( 64* ...15bytes... (0) (Right))
[00:00:00:00] [*] Found by grey-box concolic (1 new edges) 
[00:00:00:00] [*] Found by grey-box concolic (2 new edges) 
INFO:deepstate.executors.fuzz.eclipser:Performing decoding on testcases
  and crashes
...
> ./runlen --fuzz --output\_test\_dir bruteforce\_fuzzing
INFO: Starting fuzzing
WARNING: No seed provided; using 1619628901
CRITICAL: Runlen.cpp(60): ORIGINAL: 'd011', ENCODED: 'dA0A1A', ROUNDTRIP: 'd01'
ERROR: Failed: Runlength\_EncodeDecode
INFO: Saved test case in file
`bruteforce\_fuzzing/5ef02d497259a3c0bd3bda1a2f180cf5be5db973.fail`
...
\end{code}
}
\caption{Using DeepState to Test the Runlength Encoder}
\label{fig:rundeep}
\end{figure}

Replaying a test case is then trivial:

{\scriptsize
  \begin{code}
> ./runlen --input\_test\_file
    bruteforce\_fuzzing/5ef02d497259a3c0bd3bda1a2f180cf5be5db973.fail
TRACE: Initialized test input buffer with 20 bytes of data from
`bruteforce\_fuzzing/5ef02d497259a3c0bd3bda1a2f180cf5be5db973.fail`
TRACE: Running: Runlength\_EncodeDecode from Runlen.cpp(55)
CRITICAL: Runlen.cpp(60): ORIGINAL: 'd011', ENCODED: 'dA0A1A',
ROUNDTRIP: 'd01'
ERROR: Failed: Runlength\_EncodeDecode
ERROR: Test case
bruteforce\_fuzzing/5ef02d497259a3c0bd3bda1a2f180cf5be5db973.fail failed    
\end{code}
}

The directory structure for other fuzzers is more complex, but largely consistent across fuzzers.  Once users know where output files are, it is easy to, e.g., collect all test cases from all testing approaches applied.  DeepState also provides command line option to run only selected tests, if multiple tests are defined in a harness, and other common unit testing features.

Fuzzing with afl, libFuzzer, or another fuzzer requiring compile-time instrumentation is only slightly more difficult:

{\scriptsize
\begin{code}
> deepstate-afl --compile\_test ./Runlen.cpp --out\_test\_name runlen
...
> deepstate-afl ./runlen.afl --fuzzer\_out --output\_test\_dir afl\_fuzzing
...
> deepstate-libfuzzer --compile\_test ./Runlen.cpp --out\_test\_name runlen
...
> deepstate-libfuzzer ./runlen.libfuzzer --fuzzer\_out --output\_test\_dir
  libfuzzer\_fuzzing
\end{code}
}

\noindent DeepState knows the required compiler options to produce instrumented fuzzing executables.


\begin{figure}
{\scriptsize
\begin{code}
    OneOf(
	  [\&] \{MakeNewPath(path);
            r = tfs\_mkdir(sb, path);
            LOG(TRACE) << "tfs\_mkdir(sb, \"" << path << "\") = " << r;
	  \},
	  [\&] \{MakeNewPath(path);
            r = tfs\_rmdir(sb, path);
            LOG(TRACE) << "tfs\_rmdir(sb, \"" << path << "\") = " << r;
	  \},
	  [\&] \{r = tfs\_ls(sb);
            LOG(TRACE) << "tfs\_ls(sb) = " << r;
	  \},
   ...
\end{code}
}
  \caption{OneOf in Action}
  \label{fig:oneof}
  \end{figure}


  A major feature of the DeepState API not shown in the first example is the {\tt OneOf} construct, which takes as parameter a vector, array, string, or an arbitrary number of C++ lambdas.  In the first three cases, it returns a nondetermnistic element of the sequence; for lambdas, it selects and executes one of the code choices.  This makes the basic ``choose and run one option'' structure of API call sequence testing trivial to implement.  Figure~\ref{fig:oneof} shows use of OneOf in a harness for testing a file system developed at U. Toronto \cite{testfs}.  For an extended example of {\tt OneOf} see the TestFS harness (\url{https://github.com/agroce/testfs}) or a DeepState harness for differential testing of Google's leveldb and Facebook's rocksdb \url{https://github.com/agroce/testleveldb}.  Behind the scenes, the alphabet parameter to {\tt DeepState\_CstrUpToLen} also operates as a {\tt OneOf}, which means that swarm testing (see below) and other {\tt OneOf} semantics automatically are used when producing restricted-alphabet strings.  In addition to {\tt OneOf}, DeepState also provides {\tt OneOfP}, which allows users to control probabilities of options in a nondeterminstic choice.  DeepState mediates the translation, since probabilities are meaningless in pure symbolic execution, but may be very useful for fuzzers, in particular the brute-force fuzzer.


  
\section{Supported Fuzzers and Back Ends}

DeepState includes full-featured front-ends for libFuzzer, afl \cite{aflfuzz}, Eclipser \cite{choi:icse:2019}, Angora \cite{chen2018angora}, and Honggfuzz (\url{https://github.com/google/honggfuzz}), as well as a built-in, extremely fast (if dumb) fuzzer for quick discovery of bugs.  It also, using the same interface, allows users to generate tests via symbolic execution using Manticore \cite{mossberg2019manticore}, as noted above.

In addition, DeepState can interface with \emph{any} fuzzer that allows fuzzing via either file inputs or stdin, which means, in practice, the majority of fuzzers that are being developed now.  If a fuzzer is similar in interface to a fully-supported fuzzer (e.g., afl, the basis for many fuzzer variants), then writing and submitting a first-class interface is usually trivial.  Support for more fuzzers and symbolic execution tools is in progress.

\section{More Than Just a Front-End}

DeepState, in addition to allowing users to test complex properties with a variety of fuzzer and symbolic execution backends, also offers features not included in the back-end tools, including powerful methods for improving the effectiveness of tests.

\subsection{Swarm Testing}

Swarm testing \cite{ISSTA12,groce2013help} is a low-cost, high-impact approach to improving test generation, widely used in compiler testing \cite{le2014compiler,dewey2015fuzzing} and as a foundation for the test approach for FoundationDB, the back-end database for Apple and Snowflake cloud services \cite{zhou2021foundationdb}.  Heretofore, no fuzzers to our knowledge have included support for swarm testing, which operates by constructing tests that universally omit some options in repeated choices (e.g. among API calls or options).  The swarm testing concept aligns perfectly with DeepState's {\tt OneOf} construct, and DeepState allows the use of swarm testing with any fuzzer, simply by compiling the DeepState harness with the swarm option enabled.  The impact can be extremely large.  For example, finding the simple stack overflow in \url{https://github.com/agroce/deepstate-stack} requires less than one second using afl, libFuzzer, or even DeepState's built-in dumb fuzzer, using swarm testing, but an average of over an hour to detect with afl when not using swarm testing, and (we estimate) trillions of years with the brute-force fuzzer not using swarm testing.  A core goal of DeepState is to enable the exploration and use of test generation strategies that do not need to be implemented in individual fuzzers, but operate on a more semantic level.

\subsection{State-of-the-Art Test Reduction}

Fuzzer-generated tests, particularly of API call sequences, are notoriously hard to understand \cite{DD}.  DeepState therefore provides a state-of-the-art test reduction facility.  In addition to producing smaller tests that omit unnecessary steps, DeepState's reducer attempts to \emph{normalize} failing tests \cite{OneTest,MaciverD19} to simplify test values and make it easier to perform bug triage \cite{PLDI13}.

Reducing a test produced by any back-end is simple:

{\scriptsize
  \begin{code}
> deepstate-reduce ./runlen
  bruteforce\_fuzzing/5ef02d497259a3c0bd3bda1a2f180cf5be5db973.fail
  min.fail
INFO:deepstate:Setting log level from DEEPSTATE\_LOG: 2
Original test has 20 bytes
Applied 4 range conversions
Writing reduced test with 20 bytes to min.fail
=====================================================
Iteration \#1 0.01 secs / 2 execs / 0.0\% reduction
Removed 1 byte(s) @ 0: reduced test to 19 bytes
Writing reduced test with 19 bytes to min.fail
 0.02 secs / 3 execs / 5.0\% reduction
=====================================================
...
Byte range removal: PASS FINISHED IN 0.0 SECONDS, RUN: 0.11 secs / 28 execs /
  95.0\% reduction
Reduced byte 0 from 3 to 2
Writing reduced test with 1 bytes to min.fail
 0.12 secs / 31 execs / 95.0\% reduction
...
Completed 2 iterations: 0.14 secs / 36 execs / 95.0\% reduction
Padding test with 11 zeroes
Writing reduced test with 12 bytes to min.fail
> ./runlen --input\_test\_file min.fail
TRACE: Initialized test input buffer with 12 bytes of data from `min.fail`
TRACE: Running: Runlength\_EncodeDecode from Runlen.cpp(50)
CRITICAL: Runlen.cpp(55): ORIGINAL: 'aa', ENCODED: 'aA', ROUNDTRIP: 'a'
ERROR: Failed: Runlength\_EncodeDecode
ERROR: Test case min.fail failed
\end{code}
}

The original failing input ({\tt "d011"}, see Figure~\ref{fig:rundeep}) is reduced to the ``simplest'' possible failing input, {\tt "aa"}.  The criteria for reduction can be more complex than simply preserving failure, including checking an arbitrary regular expression over the failure symptom, or even running a script to ensure maintenance of code coverage.

\subsection{Ensemble Fuzzing}

Ensemble fuzzing \cite{chen2019enfuzz} extends the idea of ensemble learning, where multiple machine learning methods are applied to a problem, given the unpredictable diversity of performance of methods, to the fuzzing problem.  DeepState supports automatic cooperative fuzzing, using tests generated by one fuzzer to seed another.  The best known other implementation of ensemble fuzzing (\url{http://wingtecher.com/Enfuzz}) requires uploading source code to a web site, supports a more limited set of fuzzers than DeepState (lacking Eclipser and Angora), and requires use of the very limited API of a standard libFuzzer char buffer interface to fuzzing.  DeepState in contrast lets users write parameterized unit tests as usual, but then use any fuzzers DeepState provides for cooperative fuzzing.

\subsubsection{Experimental Evaluation}

\begin{table}
\centering
\begin{tabular}{l|r|r}
 & Runlen & TweetNaCl Bug \\
  Fuzzer & Mean Crash Time & Mean Crash Time \\
  \hline
  {\bf Ensembler} & 3.29s & 4m13s \\
  afl & 7.01s & 3m 39s \\
  libFuzzer & 4.05s & 4m 19s \\
  Angora & 7.06s & 10m 47s \\
  Eclipser & 2.74s & 12m 45s \\
\end{tabular}
\caption{Ensemble fuzzing experiments}
\label{tab:ensemble}
\end{table}

Table~\ref{tab:ensemble} shows, for simple examples, some of the benefits of the ensembler.  The first example, Runlen, is the DeepState example included in the distribution and discussed above.  The second is a real-world bignum vulnerability (\url{http://seb.dbzteam.org/blog/2014/04/28/tweetnacl_arithmetic_bug.html\#fn:2}).  In addition to the benefits described in the academic literature, where cooperative fuzzing can find bugs no single fuzzer finds well, an ensemble also mediates fuzzer variability, often (as here) ensuring that any faults are found about as quickly as by the \emph{best} of the individual fuzzers.  Note that Eclipser and Angora work well on the simple buggy run length encoder, but struggle with the more complex example.  Is it a good idea to use Eclipser?  Ensemble fuzzing lets a developer sitestep such questions.

\subsection{Work in Progress: Advanced Features}

In addition to these fully developed features, work is in progress to allow DeepState to generate standalone GoogleTest unit tests from any fuzzing-produced test (a prototype version of this functionality exists, which additionally allows afl to directly generate unit test files as it runs), to enable DeepState to check properties of the output (logging or internal printing) from test execution as in the LogScope language \cite{scriptstospecs}, and numerous other useful features that are unlikely to be added to any of DeepState's back-ends (e.g., autoamatic checks for assertions that are never executed).

\section{Related Work}

DeepState lies at the intersection of practical work on unit testing frameworks (e.g., JUnit or GoogleTest) and test generation research.  In particular, DeepState inherits from two approaches.  Parameterized unit testing \cite{ParamUnit,UnitMeister} argues that ``closed'' unit tests should be opened by providing a way to specify that some values are to be filled in by a tool.  DeepState extends this idea to allow the generation to be done via either fuzzers or symbolic execution tools, or simply by stored test cases.  Second, DeepState is fundamentally a property-based testing \cite{ClaessenH00}  tool, where specifications work like those in any property-based testing tool for an imperative language.  The idea of a universal automated test generation tool where the underlying technology is not the focus arises from efforts to test Mars Rover code at NASA/JPL \cite{AMAI,WODACommon,WODA08}.  Finally, the very recent work of Reid et al. \cite{reid2020making}, proposes the same goal as we do: to improve adoption of powerful modern testing methods by essentially allowing developers to work with paradigms with which they are familiar, though with a focus on formal verification and Rust, not fuzzing and C/C++.

\section{Conclusions}

DeepState is available as an open source tool on GitHub \url{https://github.com/trailofbits/deepstate}, and includes support for automatically building a docker environment with all supported fuzzers and symbolic execution tools installed.  The version used in the demonstration and in preparing this paper can be downloaded via {\tt docker pull agroce/deepstate\_demo:latest}.  With DeepState, every developer who knows how to write unit tests in C and C++ can take advantage of the power of modern fuzzers and symbolic execution tools.  Moreover, we believe that DeepState may be a useful basis for experiments comparing various fuzzers, as it provides a common semantics for tests, without forcing researchers to develop equivalent test harnesses.

\bibliographystyle{plain}
\bibliography{bibliography}

\newpage

\appendix
\section{Proposed Demonstration}

\begin{enumerate}

\item

  Begin with the problem:  developers could find many bugs using modern automated test generation tools, but few developers know how to use even afl, much less symbolic execution tools.

\item  Show the code for using the afl fuzzer to test a simple function that takes three ints as input, parsing the raw binary data into integers.

\item Show a Manticore script for binary analysis of a C program.

\item Say that while developers don't know how to use these tools, they DO know how to use unit testing, often.  Show the GoogleTest website and a search showing the vast number of github repos referencing GoogleTest.  Point out that there are other C/C++ unit testing frameworks, all look fairly similar.

\item Discuss that DeepState meets developers where they are:  it lets them write parameterized unit tests in C/C++ using an API similar to GoogleTest, then use various back-ends to generate tests.
  
\item

  First, show DeepState github repo, how to build deepstate docker image, compile DeepState directly on a linux/mac machine.

\item

  Launch a DeepState docker image.  Navigate to the examples directory.  Remove the Boring test from the Runlen example, explain that DeepState has options to control which test(s) run, but for demonstration purposes, we'll stick to just the one interesting test.  Step through the source code, showing the call that makes the test parameterized.

\item Compile the test at the command line, and show how to run:

  \begin{itemize}
  \item The brute-force fuzzer
  \item Eclipser
  \item The Manticore symbolic execution tool
  \end{itemize}

  Explain that manticore will be much slower than the fuzzers, but can sometimes find problems other tools cannot.  Using Manticore without DeepState is painful (even though Manticore is ``friendly'' as binary analysis tools go), this was the actual first motivation for writing DeepState!

\item Show how DeepState handles compiling code that has to be instrumented.  Run afl and libFuzzer on the same example.
  
\item Replay failing tests from the brute force fuzzer and afl.

\item Show how to use DeepState to reduce/simplify a failing test.  The brute force and afl tests reduce to the same input, making bug triage easier.

\item Show the DeepState C-Blosc2 harness, briefly step through the code, and show the bug found using DeepState.

\item Briefly mention swarm testing, show the deepstate-stack example in docker, show how with swarm it trivially finds the stack overflow, but without it, afl will need an hour or more to find the bug, the brute force fuzzer will NEVER find it (do back of the envelope math showing need for trillions of years of runtime).
  
\item Mention related work, the general goal of meeting developers where they are, invite users to play with the uploaded demo docker image.

\end{enumerate}

\end{document}